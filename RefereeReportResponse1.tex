\documentclass[11pt]{article}

\input{./preamble}

\usepackage[style=alphabetic]{biblatex}
\renewcommand{\subtitlepunct}{: }
\addbibresource{./refs.bib}
\AtNextBibliography{\small}

\begin{document}

\subsubsection*{Preliminary Remarks}

\begin{enumerate}
\item As a result of the adjustments in the most recent version, there has been a slight change in
numbering throughout the paper. For example, Theorem 4.9 has become Theorem 4.10. Unless otherwise
indicated, the numbers referred to below are the ones in the most recent version.
\end{enumerate}

\subsubsection*{Response to the Referee's Suggestions}

\begin{enumerate}
\item We have included a discussion of Definition 2.10 in the contex of stacks over the site of
affine schemes in Remark 2.12. In particular, we have clarified here how the usual moduli stack of
vector bundles can be obtained by specializing Definition 2.10.
We have also added correct references to the definition in the contexts of non-derived and
derived algebraic geometry. We mainly use the non-derived version in the paper. \item Upon the referee's encouragement, we weave a deeper narrative across the introduction that situates our programme and results within the works of King, Perroni, Craw--Smith, Abdelgadir, Porta and Porta--Sala, and others.  We believe this improves the accessibility for the reader and recognizes important contributions.

\item In the paragraph before Assumption 2.1, we have provided the origin of the assumptions of
Section 2: in particular, these assumptions are modeled after well-known properties of the
categories of quasicoherent sheaves and vector bundles on stacks over the site of affine schemes.
Furthermore, in Remark 2.16, we have clarified that these assumptions are satisfied by all stacks
over the site of affine schemes, regardless of algebraicity/geometricity. This includes
Deligne-Mumford and Artin stacks. This also includes root stacks, as they are defined as pullbacks
of stacks.

\item We have removed Conjectures 2.14 and 2.16 (previous numbering) and added Remarks 2.17 and 2.19
in their place.
The purpose of these remarks is to provide motivation for future extensions of the work to moduli
stacks in the settings of derived algebraic geometry and analytic geometry.
We should note that the assumptions in Section 2 are requirements for the results of the paper, not
the other way around.
The assumptions themselves hold for arbitrary stacks over the site of affine schemes, for formal
reasons. They do not require any form of algebraicity/geometricity or other special properties.
Thus, we expect them to hold for the analytic stacks of Clausen--Scholze or of
Ben-Bassat--Kelly--Kremnizerv --- these are just stacks over the sites of
analytic rings and simplicial commutative complete bornological rings respectively --- for similar
formal reasons.

\item We have added Remark 2.20 to point the reader to the discussion of the connection of the paper
with Nakajima quiver varieties given in Section 5.

\item We clarify the reason for considering unlabeled quivers in Remark 2.23 while indicating how
quiver varieties obtained from labeled quivers can be recovered from our work. Here, we again point
the reader to Section 5 where a more detailed discussion of these aspects is given. However, we
note that a treatment of quiver varieties is not a major purpose of this paper, and hence we have
deferred a thorough discussion of quiver varieties and labelings on quivers to future work.

\item We have made spelling corrections in various parts of the paper, including the ones pointed
out by the referee.
\end{enumerate}

\subsubsection*{Additional Updates}

\begin{enumerate}
\item While the paper has been in review, we have detected an error in the definition of the moduli
stack of vector bundle morphisms given in Section 3. This required a rewrite of
that section in the most recent version of the paper.
However, the properties of the moduli stack of vector bundle
morphisms used in the rest of the paper are unchanged even with this new definition.
As a result, we only had to make minor adjustments in Sections 4, 5 and 6.

\item We have modified the statements of Theorem 4.10 and Corollary 4.11 to organize the hypotheses
more clearly, but there is no change in the essential content of these results.

\item While this paper has been in review, Mauro Porta and Francesco Sala have pointed us to their
work constructing derived moduli stacks of diagrams of vector bundles, Higgs bundles and flat
connections, as well as showing their algebraicity. We have added remarks throughout the paper
(please see Section 1, Remark 1.1, Remark 3.4, Remark 4.12 and Remark 6.11) describing the
similarities and differences with their work.

\end{enumerate}

\end{document}
